%%%%%%%%%%%%%%%%%%%%%%%%%
%% Template for small reports
%%%%%%%%%%%%%%%%%%%%%%%%%

\documentclass[letterpaper,notitlepage,11pt]{article}

%Additional packages
\usepackage{fullpage} %For using small margins 
\usepackage{latexsym}
\usepackage{amssymb}
\usepackage{amsmath}
\usepackage[english]{babel}
\usepackage[pdftex]{color,graphicx} %Images for pdfs
\usepackage[pdftex,colorlinks]{hyperref} %Hyperlinked pdfs
\usepackage{listings} %For adding code

%If you need to add math
\newenvironment{proof}{\begin{trivlist} \item[] {\em Proof:}}{\hfill $\Box$ \end{trivlist}}
\newtheorem{theorem}     {Theorem}
\newtheorem{lemma}       {Lema}
\newtheorem{observation} {Observation}
\newtheorem{proposition} {Proposition}
\newtheorem{conjecture}  {Conjecture}
\newtheorem{corollary}   {Corolari}
\newtheorem{property}    {Property}
\newtheorem{definition}  {Definition}

%Title box
\newcommand{\titlebox}[6]
{\noindent\fbox{\parbox{\textwidth}{#1 \hfill\textbf{#2}\begin{center} 
\LARGE #3 \end{center}#4 $<$\href{mailto:#5}{#5}$>$ \hfill #6}}\bigskip\\}

%If you want to include code
%standard configuration of the listings package
%\lstset{language=matlab, basicstyle=\footnotesize, numbers=left, frame=lines, frameround=tfft,breaklines=true}
%how to include a file "\lstinputlisting{../../Courses/16720-CV/hw2/matlab/blockify.m}"


%%%%%%%%%%%%%%%%%%%%%%
%% Here begins the document
%%%%%%%%%%%%%%%%%%%%%%
\begin{document}
\titlebox
{CMU  - Robotics Institute}                           % Institution (ex: CMU - Robotics Institute)
{Manipulation Lab - Simple Hands Project}    % Center (ex: Manipulation Lab)
{Logger node - ROS - Introduction}               % Title
{}                                                                   % Author
{manipulationLab@gmail.edu}                       % e-mail
{\today}                                                         % Date (format yyyy/mm/dd)


\section{Logger}
Logger\_node is a semiautomated solution for logging the state of the
system for specific periods of time.  Logger\_node provides
functionality to log to file all topics published to the Mlab ROS
network, append information to log files and manage the location in
the computer of those same files.  The default folder where log files
are being saved is \emph{log}, relative to the global log folder
defined by the system parameter \emph{logger/folder}.

Logger\_node currently samples and logs, when available, the following
topics:

\begin{itemize}
\item robot\_CartesianLog
\item robot\_JointsLog
\item robot\_ForceLog
\item hand\_EncodersLog
\item hand\_AnglesLog
\end{itemize}

In the future it should also automatically log, when present, the
topics published by the Mocap and Kinect nodes.  The rate at which
logger samples all topics is specified by the system parameter
\emph{logger/sampligRate}.

Besides logging to file, logger\_node publishes an aggregated topic
with all relevant information of the current state of the system.  The
frequency is also specified by \emph{logger/samplingRate}.  It
reflects the exact same information being logged to file.  For this
reason, it is convenient to use this topic for data driven/learning
applications.  the aggregated log contains two timers. The first one
reflects the global ROS time.  The second one is always zero, except
if logging to file, when reflects the time since the logging procedure
stated.  The second one can be used to synchronize real time date with
previously logged data.

It is composed of two ROS packages: logger\_node, a standalone typical
ROS package that provides topics and services, and logger\_comm,
including the message definitions and a C++ client class to simplify
the invocation and communication with logger\_node.

\subsection{logger\_node}
\noindent (Located at \url{svnroot/code/nodes/logger/ROS/logger_node})
\\

\textbf{\underline{Services:}}

\begin{itemize}
\item \textbf{logger\_Start}: Service to start logging to file. 

\begin{verbatim}
int64 id         # Experiment id (added to the first line of the log file).
string folder    # Folder where to save it (relative to logFolder/log)
---
string filename  # Complete name of the logFile (relative to the logFolder)
int64 ret        # Set to 1 (success) or 0 (error)
string msg       # Error description
\end{verbatim}

\item \textbf{logger\_Stop}: Service to stop logging and close the file.

\begin{verbatim}
---
int64 ret        # Set to 1 (success) or 0 (error)
string msg       # Error description
\end{verbatim}

\item \textbf{logger\_Create}: Service to create an empty log file.

\begin{verbatim}
---
string filename    # Complete name of the logFile (relative to the logFolder)
int64 ret          # Set to 1 (success) or 0 (error)
string msg         # Error description
\end{verbatim}

\item \textbf{logger\_Append}: Service to append a string to the end
  of a log file.

\begin{verbatim}
string filename    # Name of the logFile where to append information (relative to logFolder).
string info        # String to append.
---
int64 ret          # Set to 1 (success) or 0 (error)
string msg         # Error description
\end{verbatim}

\item \textbf{logger\_Copy}: Service to copy a log file into a
  different folder.  The specified folder is always relative to the
  main logFolder.  This service can be used to copy files of a
  specific experiment into a dedicated folder.

\begin{verbatim}
string filename      # Name of the logFile to copy (realtive to the logFolder).
string folder        # Folder where to copy (relative to the logFolder)
---
int64 ret            # Set to 1 (success) or 0 (error)
string msg           # Error description
\end{verbatim}
\end{itemize}

\textbf{\underline{Topics:}}

\begin{itemize}
\item \textbf{logger\_SystemLog}: Topic that aggregates all other
  topics published to the ROS network.

\begin{verbatim}
float64 timeStamp    # ROS time-stamp
float64 logtimeStamp # Time since the beginning of last logStart.
int64 id             # Id of file begin logged.
float64 x            # x-coordinate of the robot.
float64 y            # y-coordinate of the robot.
float64 z            # z-coordinate of the robot.
float64 q0 	     # q0-coordinate of the robot.
float64 qx           # qx-coordinate of the robot.
float64 qy	     # qy-coordinate of the robot.
float64 qz	     # qz-coordinate of the robot.
float64 j1	     # joint 1 of the robot.
float64 j2	     # joint 2 of the robot.
float64 j3	     # joint 3 of the robot.
float64 j4	     # joint 4 of the robot.
float64 j5	     # joint 5 of the robot.
float64 j6	     # joint 6 of the robot.
float64 fx	     # force in the x direction.
float64 fy	     # force in the y direction.
float64 fz	     # force in the z direction.
float64 tx	     # torque in the x direction.
float64 ty	     # torque in the y direction.
float64 tz	     # torque in the z direction.
int64 encMotor	     # motor encoder.
int64[] encFinger    # array of finger encoders.
float64 angleMotor   # motor angle.
float64[] angle	     # array of finger angles.
\end{verbatim}

\end{itemize}


\subsection{logger\_comm}
\noindent (Located at \url{svnroot/code/nodes/logger/ROS/logger_comm})

\noindent Logger\_comm is a ROS wrapped C++ class meant to simplify the
communication with logger\_node. It contains:

\begin{enumerate}
\item Message and Service definitions. When creating a ROS application
  that uses logger\_node, the application only needs to be dependent
  on logger\_comm. This avoids having to link and compile against the
  entire logger\_node.
\item C++ client class LoggerComm that handles all configuration and
  initialization required to use logger\_node. As an example, if we
  want to call the service logger\_Start, we only need to instantiate
  a LoggerComm object and call to the member routine
  loggerStart. Otherwise we would have to include all required header
  files, subscribe to the logger\_Start service and format the
  required message to be sent to the service.
\end{enumerate}

\textbf{\underline{Usage example:}}

\begin{verbatim}
  ...
  ros::NodeHandle node;
 
  LoggerComm logger;           //Create LoggerComm client.
  logger.subscribe(&node);     //Subscribe to all services of logger_node.
  logger.Start();              //Start logging to file.
  ...
  logger.Stop();               //Stop logging to file.
  ...
\end{verbatim}


\textbf{\underline{Member routines}}:

\begin{verbatim}
class LoggerComm
{
  public:
    LoggerComm();
    LoggerComm(ros::NodeHandle* np);
    ~LoggerComm();

    // Subscribe Function
    void subscribe(ros::NodeHandle* np);

    // Subscribe to Topics
    void subscribeSystem(ros::NodeHandle* np, int q_len, 
        void (*funcPtr)(const logger_comm::logger_SystemLogConstPtr&));

    // Shutdown service clients
    void shutdown();

    bool Start(string &filename, string folder, int id=0);
    bool Start(string &filename, int id=0);
    bool StartGrasp(string &filename, int id=0);
    bool StartPlace(string &filename, int id=0);
    bool Start(int id=0);
    bool StartGrasp(int id=0);
    bool StartPlace(int id=0);
    bool Stop();
    bool Append(string filename, string info);
    bool Create(string &filename);
    bool Copy(string filename, string folder);
...
};
\end{verbatim}


%%%%%%%%%%%%%%%%%%%%
%% If you want bibliography
%%%%%%%%%%%%%%%%%%%%
%\bibliographystyle{unsrt}
%\bibliography{} %Name and location of the bibfile

%To cite a reference -> \cite{cite_name}
%To make a reference appear in the bibliography without a citation in the text -> \nocite{cite_name}

\end{document}
