%%%%%%%%%%%%%%%%%%%%%%%%%
%% Template for small reports
%%%%%%%%%%%%%%%%%%%%%%%%%

\documentclass[letterpaper,notitlepage,11pt]{article}

%Additional packages
\usepackage{fullpage} %For using small margins 
\usepackage{latexsym}
\usepackage{amssymb}
\usepackage{amsmath}
\usepackage[english]{babel}
\usepackage[pdftex]{color,graphicx} %Images for pdfs
\usepackage[pdftex,colorlinks]{hyperref} %Hyperlinked pdfs
\usepackage{listings} %For adding code

%If you need to add math
\newenvironment{proof}{\begin{trivlist} \item[] {\em Proof:}}{\hfill $\Box$ \end{trivlist}}
\newtheorem{theorem}     {Theorem}
\newtheorem{lemma}       {Lema}
\newtheorem{observation} {Observation}
\newtheorem{proposition} {Proposition}
\newtheorem{conjecture}  {Conjecture}
\newtheorem{corollary}   {Corolari}
\newtheorem{property}    {Property}
\newtheorem{definition}  {Definition}

%Title box
\newcommand{\titlebox}[6]
{\noindent\fbox{\parbox{\textwidth}{#1 \hfill\textbf{#2}\begin{center} 
\LARGE #3 \end{center}#4 $<$\href{mailto:#5}{#5}$>$ \hfill #6}}\bigskip\\}

%If you want to include code
%standard configuration of the listings package
%\lstset{language=matlab, basicstyle=\footnotesize, numbers=left, frame=lines, frameround=tfft,breaklines=true}
%how to include a file "\lstinputlisting{../../Courses/16720-CV/hw2/matlab/blockify.m}"


%%%%%%%%%%%%%%%%%%%%%%
%% Here begins the document
%%%%%%%%%%%%%%%%%%%%%%
\begin{document}
\titlebox
{CMU  - Robotics Institute}                           % Institution (ex: CMU - Robotics Institute)
{Manipulation Lab - Simple Hands Project}    % Center (ex: Manipulation Lab)
{Matlab node - ROS - Introduction}               % Title
{}                                                                   % Author
{manipulationLab@gmail.edu}                       % e-mail
{\today}                                                         % Date (format yyyy/mm/dd)


\section{Matlab}
Matlab\_node is a ROS interface to the MATLAB engine. It allows to
execute commands in the MATLAB command line as well as inserting and
recover variables from the MATLAB workspace.  When executed,
matlab\_node opens a MATLAB instance in background and without a
graphical interface.

It is composed of two ROS packages: matlab\_node, a standalone typical
ROS package that provides topics and services, and matlab\_comm,
including the message definitions and a C++ client class to simplify
the invocation and communication with matlab\_node.

Notes on installation of Matlab engine and how to configure
the computer to execute matlab\_node can be found at
\url{http://simplehands.wikispaces.com/Matlab+Installation}.

\subsection{matlab\_node}
\noindent (Located at \url{svnroot/code/nodes/matlab/ROS/matlab_node})
\\

\textbf{\underline{Services:}}

\begin{itemize}
\item \textbf{matlab\_Ping}: Service to ping the matlab interface. It
  returns true only when the MATLAB engine is opened and all services
  are ready to serve petitions.

\begin{verbatim}
---
int64 ret      # Set to 1 (success) or 0 (error)
string msg     # Error description
\end{verbatim}

\item \textbf{matlab\_SendCommand}: Service to execute a command in
  the MATLAB command line. It the command is a function that returns
  something, we can recover the return value by saving it to a variable
  in the executed command and recover that variable with the
  appropriate service.

\begin{verbatim}
string command    # Command to be executed in the MATLAB command line
---
int64 ret         # Set to 1 (success) or 0 (error)
string msg        # Error description
\end{verbatim}

\item \textbf{matlab\_GetArray}: Service to get an array from the
  MATLAB workspace.

\begin{verbatim}
string name       # Name of the variable to recover. It should be an array.
---
int64 ncols       # Recovered number of columns of the array.
int64 nrows       # Recovered number of rows of the array.
float64[] data    # Data recovered
int64 ret         # Set to 1 (success) or 0 (error)
string msg        # Error description
\end{verbatim}

\item \textbf{matlab\_GetString}: Service to get a string from the
  MATLAB workspace.

\begin{verbatim}
string name       # Name of the variable to recover. It should be an array.
---
string data       # Data recovered
int64 ret         # Set to 1 (success) or 0 (error)
string msg        # Error description
\end{verbatim}

\item \textbf{matlab\_PutArray}: Service to put an array in the
  MATLAB workspace.

\begin{verbatim}
int64 nrows 	  # Number of rows of the array
int64 ncols 	  # Number of columns of the array
float64[] data    # Data (row by row)
string name       # Name to give to the array in the MATLAB workspace
---
int64 ret         # Set to 1 (success) or 0 (error)
string msg        # Error description
\end{verbatim}

\item \textbf{matlab\_PutString}: Service to put an string in the
  MATLAB workspace.

\begin{verbatim}
string name       # Name to give to the string in the MATLAB workspace
string data       # String data
---
int64 ret         # Set to 1 (success) or 0 (error)
string msg        # Error description
\end{verbatim}
\end{itemize}


\subsection{matlab\_comm}
\noindent (Located at \url{svnroot/code/nodes/matlab/ROS/matlab_comm})

\noindent Matlab\_comm is a ROS wrapped C++ class meant to simplify the
communication with matlab\_node. It contains:

\begin{enumerate}
\item Message and Service definitions. When creating a ROS application
  that uses matlab\_node, the application only needs to be dependent
  on matlab\_comm. As a consequence, the application does not need to
  link to all the matlab engine drivers and libraries. The computer
  executing the ROS application does not need to have MATLAB installed.
\item C++ class MatlabComm that handles all configuration and
  initialization required to use matlab\_node. As an example, if we want
  to call the service matlab\_Ping, we only need to instantiate a MatlabComm
  object and call to the member routine matlabPing. Otherwise
  we would have to include all required header files, subscribe to the
  matlab\_Ping service and format the required message to be sent to the
  service.
\end{enumerate}

\textbf{\underline{Usage example:}}

\begin{verbatim}
  ...
  ros::NodeHandle node;
 
  MatlabComm matlab;           //Create MatlabComm client.
  matlab.subscribe(&node);     //Subscribe to all services of matlab_node.
  while(!matlab.Ping());       //Wait until MATLAB is ready.
  matlab.SendCommand("a=3;");  //Send a command.

  ...
\end{verbatim}


\textbf{\underline{Member routines}}:

\begin{verbatim}

class MatlabComm
{
  public:
    MatlabComm();
    MatlabComm(ros::NodeHandle* np);
    ~MatlabComm();

    // Subscription
    void subscribe(ros::NodeHandle* np);

    // Call this before program exits so we don't have double freeing issues
    void shutdown();
    
    //Client functions to simplify calling Matlab ROS services
    bool Ping();
    
    //send
    bool sendCommand(const char *command);
    bool sendMat(const char *name, Mat &m);
    bool sendVec(const char *name, Vec &v);
    bool sendValue(const char *name, double v);
    bool sendString(const char *name, char *str);

    //receive
    bool getMat(const char *name, Mat &m);
    Mat getMat(const char *name);
    bool getVec(const char *name, Vec &v);
    Vec getVec(const char *name);
    double getValue(const char *name);
    bool getValue(const char *name, double *v);
    bool getString(const char *name, char *str, int strLength);
    std::string getString(const char *name);
...
};
\end{verbatim}


%%%%%%%%%%%%%%%%%%%%
%% If you want bibliography
%%%%%%%%%%%%%%%%%%%%
%\bibliographystyle{unsrt}
%\bibliography{} %Name and location of the bibfile

%To cite a reference -> \cite{cite_name}
%To make a reference appear in the bibliography without a citation in the text -> \nocite{cite_name}

\end{document}
