\documentclass{article}

\usepackage{fancyhdr} % Required for custom headers
\usepackage{lastpage} % Required to determine the last page for the footer
\usepackage{extramarks} % Required for headers and footers
\usepackage{graphicx} % Required to insert images
\usepackage{lipsum} % Used for inserting dummy 'Lorem ipsum' text into the template
\usepackage{ulem} % Required for Cross out text
\usepackage{amsmath} % Required for Math Stuff
\usepackage{amsfonts}
\usepackage{enumerate}
%\usepackage[]{algorithm2e}
\usepackage{hyperref}
%\usepackage{qtree}
\usepackage{hyperref}

\usepackage{caption}
\usepackage{subcaption}

% Margins
\topmargin=-0.45in
\evensidemargin=0in
\oddsidemargin=0in
\textwidth=6.5in
\textheight=9.0in
\headsep=0.25in
\setlength\parindent{0pt}

\linespread{1.1} % Line spacing

%----------------------------------------------------------------------------------------

\begin{document}

\title{Notes on regrasping as an underactuated pivot}
\author{Annie}
\maketitle
%%\vspace{-0.7in}

%----------------------------------------------------------------------------------------
%	Math Commands
%----------------------------------------------------------------------------------------

\providecommand{\abs}[1]{\left\vert#1\right\vert}
\providecommand{\norm}[1]{\left\Vert#1\right\Vert}

\section{Introduction}

-explain extrinsic regrasping

-relate to manipulation as a whole


\section{Related Work}

\section{Problem Scope}

-concept (grippers as axis of rotation,etc)

-assumptions

-how to describe goal pose

-defining stable poses

-limitations/problems/places where this wont work


\section{Grasping}

-explain how grasp is how we utilize gravity

-explain how we first find ``ideal'' grasp then adjust until feasible

\subsection{Ideal Grasp}

The basic heuristic used to find an initial grasp is based on a rotation due to gravity into the goal orientation. 

\subsection{Grasp Feasibility}

\section{Arm Motion}

\subsection{Elliptical Trammel Model}

\subsection{Pendulum Model}

\subsection{Expected Error}

\section{Implementation}






\subsection{Equations of motion}

Treating the attached object as an unactuated pendulum and the end
effector moving in a 2 dimensional plane, we have three degrees of
freedom. (We restrict the motion of the end effector to a plane simply
because motion parallel to the axis of rotation would have no effect
on the pendulum.) We will use $\mathbf{q} = [x,y,\theta]^T$ and
$\mathbf{x} =[\mathbf{q},\mathbf{\dot{q}}]^T$. $\theta$ measures the
counter-clockwise angle of the pendulum and is zero hanging straight
down. $x$ is the horizontal position of the hand, and $y$ is the
vertical position. We will denote the velocity of the cart as $\mathbf{v} = [\dot{x},\dot{y}]^T$. \\

The kinematics of the system are given by 

\begin{equation}\label{eq:state}
\mathbf{q_{cart}} = \begin{pmatrix} \dot{x} \\ \dot{y} \end{pmatrix}, \mathbf{q_{pendulum}} = \begin{pmatrix} \dot{x+l\sin(\theta)} \\ \dot{y-l\cos(\theta)} \end{pmatrix}
\end{equation}

The kinematic and potential energies are given by, respectively, 

\begin{equation}\label{eq:kinetic}
T = \frac{1}{2} (m_c + m_p) \mathbf{v} \cdot \mathbf{v} + m_p\dot{x}\dot{\theta}l\cos(\theta) + \frac{1}{2}m_pl^2\dot{\theta}^2
\end{equation}
\begin{equation}\label{eq:potential}
U = -m_pg(y+l\cos(\theta))
\end{equation}

The Lagrangian yields the equations of motion:

\begin{equation}\label{eq:x_motion}
(m_c+m_p)\ddot{x} + m_pl\ddot{\theta}\cos(\theta) - m_pl\dot{\theta}\sin(\theta) = f
\end{equation}
\begin{equation}\label{eq:y_motion}
(m_c+m_p)\ddot{y} + m_pl\ddot{\theta}\sin(\theta) + m_pl\dot{\theta}\cos(\theta) - mg = h
\end{equation}
\begin{equation}\label{eq:theta_motion}
m_pl\ddot{x}\cos(\theta) + m_pl\ddot{y}\sin(\theta) + m_pl^2\ddot{\theta} + m_pgl\sin(\theta) = -b\dot{\theta}
\end{equation}

where $\mathbf{u} = f$ and $\mathbf{v} = h$ are the forces inputted in the $\hat{x}$ and $\hat{y}$ directions, and $b$ is the coefficient of friction around the axis of rotation. \\

However, we can not continue to treat this as a simple underactuated
system similar to the cart-pole system. Since the pendulum is actually
an object held by the gripper, we have no feedback for $\theta$ to
control the system with. So our goal is to find the best open-loop
trajectory for the arm to ensure that the object swings up and there
is enough time to ``catch'' the object by closing the grippers.


The energy of the pendulum is given by 

To pump energy into the pendulum to swing it up we want the energy to be increasing
\begin{equation}\label{eq:pendE}
E = \frac{1}{2}m_pl^2\dot{\theta}^2 - m_pg(y+l\cos(\theta))
\end{equation}
\begin{equation}\label{eq:pendEdot}
\dot{E} = m_pl^2\dot{\theta}\ddot{\theta} - m_pg(\dot{y}+l\sin(\theta)\dot{\theta})
\end{equation}

\section{Grip Force}

\section{Experimental Results}

- Description of the robot, hand, fingertips, blocks, etc
- Success rates, etc

\section{Discussion}

\section{Conclusion}

\end{document}
