\documentclass{article}

\usepackage{fancyhdr} % Required for custom headers
\usepackage{lastpage} % Required to determine the last page for the footer
\usepackage{extramarks} % Required for headers and footers
\usepackage{graphicx} % Required to insert images
\usepackage{lipsum} % Used for inserting dummy 'Lorem ipsum' text into the template
\usepackage{ulem} % Required for Cross out text
\usepackage{amsmath} % Required for Math Stuff
\usepackage{amsfonts}
\usepackage{enumerate}
%\usepackage[]{algorithm2e}
\usepackage{hyperref}
%\usepackage{qtree}
\usepackage{hyperref}

\usepackage{caption}
\usepackage{subcaption}

% Margins
\topmargin=-0.45in
\evensidemargin=0in
\oddsidemargin=0in
\textwidth=6.5in
\textheight=9.0in
\headsep=0.25in
\setlength\parindent{0pt}

\linespread{1.1} % Line spacing

%----------------------------------------------------------------------------------------

\begin{document}

\title{Notes on regrasping as an underactuated pivot}
\author{Annie}
\maketitle
%%\vspace{-0.7in}

%----------------------------------------------------------------------------------------
%	Math Commands
%----------------------------------------------------------------------------------------

\providecommand{\abs}[1]{\left\vert#1\right\vert}
\providecommand{\norm}[1]{\left\Vert#1\right\Vert}

\section{Introduction}

\section{Related Work}

\section{Problem Definition}

\section{Grasping}

\section{Arm Motion}

\subsection{Equations of motion}

Treating the attached object as an unactuated pendulum and the end
effector moving in a 2 dimensional plane, we have three degrees of
freedom. (We restrict the motion of the end effector to a plane simply
because motion parallel to the axis of rotation would have no effect
on the pendulum.) We will use $\mathbf{q} = [x,y,\theta]^T$ and
$\mathbf{x} =[\mathbf{q},\mathbf{\dot{q}}]^T$. $\theta$ measures the
counter-clockwise angle of the pendulum and is zero hanging straight
down. $x$ is the horizontal position of the hand, and $y$ is the
vertical position. We will denote the velocity of the cart as $\mathbf{v} = [\dot{x},\dot{y}]^T$. \\

The kinematics of the system are given by 

\begin{equation}\label{eq:state}
\mathbf{q_{cart}} = \begin{pmatrix} \dot{x} \\ \dot{y} \end{pmatrix}, \mathbf{q_{pendulum}} = \begin{pmatrix} \dot{x+l\sin(\theta)} \\ \dot{y-l\cos(\theta)} \end{pmatrix}
\end{equation}

The kinematic and potential energies are given by, respectively, 

\begin{equation}\label{eq:kinetic}
T = \frac{1}{2} (m_c + m_p) \mathbf{v} \cdot \mathbf{v} + m_p\dot{x}\dot{\theta}l\cos(\theta) + \frac{1}{2}m_pl^2\dot{\theta}^2
\end{equation}
\begin{equation}\label{eq:potential}
U = -m_pg(y+l\cos(\theta))
\end{equation}

The Lagrangian yields the equations of motion:

\begin{equation}\label{eq:x_motion}
(m_c+m_p)\ddot{x} + m_pl\ddot{\theta}\cos(\theta) - m_pl\dot{\theta}\sin(\theta) = f
\end{equation}
\begin{equation}\label{eq:y_motion}
(m_c+m_p)\ddot{y} + m_pl\ddot{\theta}\sin(\theta) + m_pl\dot{\theta}\cos(\theta) - mg = h
\end{equation}
\begin{equation}\label{eq:theta_motion}
m_pl\ddot{x}\cos(\theta) + m_pl\ddot{y}\sin(\theta) + m_pl^2\ddot{\theta} + m_pgl\sin(\theta) = -b\dot{\theta}
\end{equation}

where $\mathbf{u} = f$ and $\mathbf{v} = h$ are the forces inputted in the $\hat{x}$ and $\hat{y}$ directions, and $b$ is the coefficient of friction around the axis of rotation. \\

However, we can not continue to treat this as a simple underactuated
system similar to the cart-pole system. Since the pendulum is actually
an object held by the gripper, we have no feedback for $\theta$ to
control the system with. So our goal is to find the best open-loop
trajectory for the arm to ensure that the object swings up and there
is enough time to ``catch'' the object by closing the grippers.


The energy of the pendulum is given by 

To pump energy into the pendulum to swing it up we want the energy to be increasing
\begin{equation}\label{eq:pendE}
E = \frac{1}{2}m_pl^2\dot{\theta}^2 - m_pg(y+l\cos(\theta))
\end{equation}
\begin{equation}\label{eq:pendEdot}
\dot{E} = m_pl^2\dot{\theta}\ddot{\theta} - m_pg(\dot{y}+l\sin(\theta)\dot{\theta})
\end{equation}

\section{Grip Force}

\section{Experimental Results}

- Description of the robot, hand, fingertips, blocks, etc
- Success rates, etc

\section{Discussion}

\section{Conclusion}


%% \begin{figure}[h!]
%%   \centering
%%   \includegraphics[width=0.75\textwidth]{diagram.png}
%%   \caption{A rough sketch of the system. The hand is in gray with gripper coming down and the block in red.}
%% \end{figure}

%% As a first pass at the problem of pivoting a block into any angle using the dynamics of the system, I am approaching it as an underactuated system where the grasp acts as a pivot joint with a certain rotation damping due to the friction of the grasp, $b$. The block has a mass $m$, moment of inertia about the block's center of mass $I$, and a length $l$. The hand has a mass, $M$ and its can move in the $\hat{i}$ and $\hat{j}$ directions as functions  $f(t)$ and $g(t)$. 

%% \section{Generalized Coordinate System and Forces}

%% There are three degrees of freedom here (assuming no motion in the $\hat{k}$ is relevant), which are x,y, and $\theta$. Positive x points to the right, y up, and $\theta$ clockwise measured from the downward position.  
%% $$
%% \zeta_j: x,y, \theta
%% $$
%% with the variation $\zeta_j: \delta x, \delta y, \delta \theta$

%% The generalized forces, $\Xi_j$ can be obtained fromt the work where
%% $$
%% \delta W = \overset{n}{\underset{i = 1}{\sum}} \mathbf{F_i} \cdot \delta \mathbf{r_i} = \overset{n}{\underset{j = 1}{\sum}} \Xi_j \delta \zeta_j
%% $$

%% Nonconservative forces from the inputs $g(t)$ and $f(t)$ and the damping $b$ around the grasp result in,
%% $$
%% \delta W = f(t) \delta x + g(t) \delta y - b \dot{\theta}\delta\theta
%% $$ 
%% So we get:
%% $$
%% \Xi_x = f(t) \\
%% \Xi_y = g(t) \\
%% \Xi_{\theta} = -b\dot{\theta}
%% $$

%% \section{Kinetic Energy}

%% The kinetic energy of the hand (the cart in simple cases): 
%% $$
%% K_{hand} = \frac{1}{2}M\mathbf{v}^2
%% $$

%% Where $M$ is the mass of the hand and v is a planar velocity vector of the hand (lets just assume x,y for now since movement in the same axis as the pivot axis doesnt do anything). \\

%% The kinetic energy for the block:
%% $$
%% K_{block} = \frac{1}{2} m\mathbf{v_c}^2 + \frac{1}{2} I \omega^2
%% $$

%% where $v_c$ is the velocity of the center of mass of the block, $m$ is the mass of the block, $I$ is the moment of inertia around the block's center of mass, and $\omega$ is the angular velocity. \\
%% The position vector can then be written,
%% $$
%% \mathbf{r_c} = (x - l\sin(\theta)) \hat{i} + (y - l\cos(\theta)) \hat{j}
%% $$
%% $$
%% \mathbf{v_c} = \frac{d\mathbf{r_c}}{dt} = (\dot{x} - l \cos(\theta)\dot{\theta}) \hat{i} + (\dot{y} + l\sin(\theta)\dot{\theta} \hat{j}
%% $$

%% Then, $\omega = \dot{\theta}$ \\
%% So plugging in we get, 
%% $$
%% K_{block} = \frac{1}{2} m (\dot{x}^2 - 2 \dot{x}lcos(\theta)\dot{\theta} + l^2cos^2(\theta)\dot{\theta}^2 + \dot{y}^2 + 2 \dot{y}lsin(\theta)\dot{\theta} + l^2sin^2(\theta)\dot{\theta}^2) + \frac{1}{2} I \dot{\theta}^2
%% $$
%% $$
%%  =  \frac{1}{2} m (\dot{x}^2 - 2 \dot{x}l\cos(\theta)\dot{\theta} + \dot{y}^2 + 2 \dot{y}l\sin(\theta)\dot{\theta} + l^2\dot{\theta}^2) + \frac{1}{2} I \dot{\theta}^2
%% $$
%% Then our total kinetic energy is, 
%% $$
%% K = K_{hand} + K_{block} = \frac{1}{2} M(\dot{x}^2+\dot{y}^2) + \frac{1}{2} m (\dot{x}^2 - 2 \dot{x}l\cos(\theta)\dot{\theta} + \dot{y}^2 + 2 \dot{y}l\sin(\theta)\dot{\theta} + l^2\dot{\theta}^2) + \frac{1}{2} I \dot{\theta}^2
%% $$

%% \section{Potential Energy}

%% $$
%% P = -mg(y+l)cos(\theta)
%% $$

%% \section{Lagrangian}

%% Fromt he kinetic and potential energies, the lagrangians is given by
%% $$
%% L = K - P
%% $$
%% which becomes
%% $$
%% L = \frac{1}{2} M(\dot{x}^2+\dot{y}^2) + \frac{1}{2} m (\dot{x}^2 - 2 \dot{x}l\cos(\theta)\dot{\theta} + \dot{y}^2 + 2 \dot{y}l\sin(\theta)\dot{\theta} + l^2\dot{\theta}^2) + \frac{1}{2} I \dot{\theta}^2 + mg(y+l)\cos(\theta)
%% $$

%% \subsection{More Lagrange Shit}

%% The equations for x,
%% $$
%% \Xi_x = \frac{d}{dt} (\frac{\partial L}{\partial \dot{x}} ) - \frac{\partial L}{\partial x}
%% $$
%% $$
%% \frac{d}{dt} (M\dot{x} + m\dot{x} - ml\cos(\theta)\dot{\theta}) = f(t) 
%% $$
%% $$
%% (M+m)\ddot{x} - ml\cos(\theta)\ddot{\theta} + ml\sin(\theta)\dot{\theta}^2 = f(t) 
%% $$

%% The same for y, 
%% $$
%% \Xi_y = \frac{d}{dt} (\frac{\partial L}{\partial \dot{y}} ) - \frac{\partial L}{\partial y}
%% $$
%% $$
%% \frac{d}{dt} (M\dot{y} + m\dot{y} + ml\sin(\theta)\dot{\theta}) - mg\cos(\theta) = g(t) 
%% $$
%% $$
%% (M+m)\ddot{y} + ml\sin(\theta)\ddot{\theta} - ml\cos(\theta)\dot{\theta}^2 - mg\cos(\theta) = g(t)
%% $$

%% And for $\theta$,
%% $$
%% \Xi_{\theta} = \frac{d}{dt} (\frac{\partial L}{\partial \dot{\theta}} ) - \frac{\partial L}{\partial \theta}
%% $$
%% $$
%% \frac{d}{dt} (-m\dot{x}l\cos(\theta) + m\dot{y}l\sin(\theta) + ml^2\dot{\theta} + I\dot{\theta}) - (ml\dot{x}\sin(\theta)\dot{\theta} + ml\dot{y}\cos(\theta)\dot{\theta} - mg\sin(\theta)) = -b \dot{\theta}
%% $$
%% $$
%% (ml^2 + I)\ddot{\theta} -ml\ddot{x}\cos(\theta)+ml\dot{x}\sin(\theta)\dot{\theta}+ml\ddot{y}\sin(\theta)+ml\dot{y}\cos(\theta)\dot{\theta} - (ml\dot{x}\sin(\theta)\dot{\theta} + ml\dot{y}\cos(\theta)\dot{\theta} - mg\sin(\theta))= -b \dot{\theta}
%% $$
%% where $b$ is a rotational damping coefficient around the grasp. \\

%% Now we have to solve for $f(t)$ (motion in the $\hat{i}$ direction) and $g(t)$ (motion in the $\hat{j}$ direction)

%% \begin{equation} \label{eq:x}
%% (M+m)\ddot{x} - ml\cos(\theta)\ddot{\theta} + ml\sin(\theta)\dot{\theta}^2 = f(t) 
%% \end{equation}
%% \begin{equation} \label{eq:y}
%% (M+m)\ddot{y} + ml\sin(\theta)\ddot{\theta} - ml\cos(\theta)\dot{\theta}^2 - mg\cos(\theta) = h(t)
%% \end{equation}
%% \begin{equation} \label{eq:theta}
%% (ml^2 + I)\ddot{\theta} + b \dot{\theta} -ml\ddot{x}\cos(\theta) +ml\ddot{y}\sin(\theta) + mg\sin(\theta)= 0
%% \end{equation}

%% \section{Energy Regulation using PFLs}

%% \subsection{Collocated Partial Feedback Linearization (PFL)}

%% Our goal is for $\ddot{x} = \ddot{x_d}$ and $\ddot{y} = \ddot{y_d}$. We can solve for $\ddot{\theta}$ using ~\ref{eq:theta}
%% $$
%% \ddot{\theta} = \frac{1}{ml^2} (-mgs -mls\ddot{y} +mlc\ddot{x} -b\dot{\theta})
%% $$
%% Where $s = \sin(\theta)$ and $c= \cos(\theta)$. \\
%% \noindent
%% If we plug this into ~\ref{eq:y} 

%% $$
%% (M+m)\ddot{y} + mls(\frac{1}{ml^2} (-mgs -mls\ddot{y} +mlc\ddot{x} -b\dot{\theta})) - mlc\dot{\theta}^2 - mgc = h(t)
%% $$
%% $$
%% (M+m)\ddot{y} + \frac{s}{l} (-mgs -mls\ddot{y} +mlc\ddot{x} -b\dot{\theta}) - mlc\dot{\theta}^2 - mgc = h(t)
%% $$
%% $$
%% \ddot{y} = \frac{1}{M+mc^2} (h(t)+\frac{mgs^2}{l}-msc\ddot{x}+\frac{bs\dot{\theta}}{l}+mlc\dot{\theta}^2+mgc)
%% $$
%% Then into ~\ref{eq:x} 
%% $$
%% (M+m)\ddot{x} - ml\cos(\theta)\ddot{\theta} + ml\sin(\theta)\dot{\theta}^2 = f(t) 
%% $$
%% etc. etc. etc.

%% \subsection{Energy Regulation}

%% If $\ddot{x} = \bar{u}$ and $\ddot{y} = \bar{v}$, then 
%% $$
%% \ddot{\theta} = \frac{1}{ml^2} (-mgs -mls\bar{v} +mlc\bar{u} -b\dot{\theta})
%% $$

%% From before our energy is given by,
%% $$
%% E =  \frac{1}{2} ml^2\dot{\theta}^2 -  mg(y+l)c
%% $$
%% $$
%% \dot{E} = ml^2 \dot{\theta}\ddot{\theta} + mg(y+l)s\dot{\theta}
%% $$
%% $$
%% \dot{E} = ml^2 \dot{\theta}(\frac{1}{ml^2} (-mgs -mls\bar{v} +mlc\bar{u} -b\dot{\theta})) + mg(y+l)s\dot{\theta}
%% $$
%% $$
%% \dot{E} = \dot{\theta}(-mgs -mls\bar{v} +mlc\bar{u} -b\dot{\theta}) + mg(y+l)s\dot{\theta}
%% $$
%% So to increase the energy, we can chose and $\bar{u}$ and $\bar{v}$ to make this expression positive. For example,

%% $$
%% \bar{u} = kmlc\dot{\theta} \bar{E}
%% $$
%% $$
%% \bar{v} = -kmls\dot{\theta} \bar{E} 
%% $$

%% Where $\bar{E}$ is the difference between the desired energy and the actual energy.

%% \subsection{now wat}

%% We have two problems: we have no feedback for $\dot{\theta}$, and
%% since our desired energy is not at an equilibrium point, it won't have
%% zero kinetic energy, so we don't really know what $\bar{E}$ is. \\ \\
%% \noindent
%% The first is easier to deal with. If we just put in just enough energy
%% to get to $\theta$ before it begins decreasing, we will be at zero
%% velocity at the apex of a path arc at $\theta$. So our $E_d =
%% mg(y+l)\cos(\theta)$. \\ \\
%% \noindent 
%% To handle no feedback (which is inherent in this problem since the
%% block is always connected at the fingertips where there is no sensing
%% normally), we can predict what $\theta$ is ideally doing based on the
%% equations of motion. \\ \\
%% \noindent
%% Let's plug our expression for $\ddot{\theta}$ into ~\ref{eq:x} (or we could use ~\ref{eq:y})
%% $$
%% \ddot{\theta} = \frac{1}{ml^2} (-mgs -mls\bar{v} +mlc\bar{u} -b\dot{\theta})
%% $$
%% We know that $f(t) = \bar{u}$ and that $\ddot{x} = \frac{f(t)}{m} = klc\dot{\theta}$. Similarly for $\ddot{y}$
%% $$
%% \ddot{\theta} = \frac{1}{ml^2} (-mgs -mls(-kls\dot{\theta}) +mlc(klc\dot{\theta}) -b\dot{\theta})
%% $$
%% $$
%% \ddot{\theta} = \frac{1}{ml^2} (-mgs +kml^2\dot{\theta} -b\dot{\theta})
%% $$
%% \noindent
%% From ~\ref{eq:x}, $(M+m)\ddot{x} - ml\cos(\theta)\ddot{\theta} + ml\sin(\theta)\dot{\theta}^2 = f(t) $

%% $$
%% (M+m)(klc\dot{\theta}) - ml\cos(\theta)(\frac{1}{ml^2} (-mgs +kml^2\dot{\theta} -b\dot{\theta})) + ml\sin(\theta)\dot{\theta}^2 = kmlc\dot{\theta}
%% $$
%% \begin{equation} \label{eq:t1}
%% (M-m)(klc\dot{\theta}) + \frac{cb\dot{\theta}}{l} + \frac{mgsc}{l} + mls\dot{\theta}^2 = 0
%% \end{equation}
%% \noindent
%% From ~\ref{eq:y}, $(M+m)\ddot{y} + ml\sin(\theta)\ddot{\theta} - ml\cos(\theta)\dot{\theta}^2 - mg\cos(\theta) = h(t)$

%% $$
%% (M+m)(-kls\dot{\theta}) + ml\sin(\theta)(\frac{1}{ml^2} (-mgs +kml^2\dot{\theta} -b\dot{\theta})) - ml\cos(\theta)\dot{\theta}^2 - mg\cos(\theta) = -kmls\dot{\theta}
%% $$
%% \begin{equation} \label{eq:t2}
%% (m-M)(kls\dot{\theta}) - \frac{bs\dot{\theta}}{l} -\frac{mgs^2}{l} - mlc\dot{\theta}^2 - mgc = 0
%% \end{equation}
%% \noindent
%% To solve for $\theta$ and $\dot{\theta}$, we can multiply ~\ref{eq:t1} by $\cos(\theta)$, ~\ref{eq:t2} by $\sin(\theta)$, and add.

%% $$
%% (M-m)(klc^2\dot{\theta}) + \frac{c^2b\dot{\theta}}{l} + \frac{mgsc^2}{l} + mlsc\dot{\theta}^2 = 0
%% $$
%% $$
%% (m-M)(kls^2\dot{\theta}) - \frac{bs^2\dot{\theta}}{l} -\frac{mgs^3}{l} - mlsc\dot{\theta}^2 - mgsc = 0
%% $$
%% \noindent
%% Giving
%% $$
%% (M-m)(kl\dot{\theta}) + \frac{b\dot{\theta}}{l} + \frac{mgs}{l} + mgsc = 0
%% $$
%% \noindent
%% Then we can solve for $\dot{\theta}$ and integrate to get $\theta$
%% $$
%% \dot{\theta} = \frac{1}{(M-m)kl + \frac{b}{l}}(-mgsc - \frac{mgs}{l})
%% $$
%% $$
%% \theta = \frac{1}{(M-m)kl + \frac{b}{l}} (-\frac{mgc^2}{2} +\frac{mgc}{l})
%% $$
%% \noindent
%% If we substitute these expressions back into our control laws $\bar{u}$ and $\bar{v}$, we get
%% $$
%% \bar{u} =  \frac{kml}{(M-m)kl + \frac{b}{l}} * c(-mgsc - \frac{mgs}{l}))
%% $$
%% $$
%% \bar{v} = \frac{-kml}{(M-m)kl + \frac{b}{l}} *  s(-mgsc - \frac{mgs}{l}))
%% $$

%% \noindent
%% Ignoring costants those expressions roughly look like
%% $$
%% \bar{u} = \cos(\theta) * (-\sin(\theta)\cos(\theta)-\sin(\theta))
%% $$
%% $$
%% \bar{v} = \sin(\theta) * (-\sin(\theta)\cos(\theta)-\sin(\theta))
%% $$

%% \begin{figure}[h!]
%%   \centering
%%   \includegraphics[width=0.75\textwidth]{curve2.png}
%%   \caption{}
%% \end{figure}

%%  integral (-cos(x)+sqrt(cos^2(x) - (1-5*sin^2(x)))))/(2*sin(x)) dx
%% cos(x*sqrt(sin^2(x))*csc(x) - (1/2)*log(sin(x)))*(-cos(x)+sqrt(cos^2(x) - (1-5*sin^2(x)))))/(2*sin(x)) 

%% We know our energy must be at least $ E \ge
%% mg(y+l)\cos(\theta)$. Maybe we can approximate the energy and it will
%% be good enough. After all we are only trying to swing the object so
%% that it falls into the region where the desired face is the support
%% face (capture region?). We do want to overshot that minimum energy
%% (also to make up for time to close the fingers). Let's use the velocity of the the hand, which we know, $\mathbf{v} = \begin{pmatrix} \dot{u} \\ \dot{v} \end{pmatrix} $ to get 

%% $$
%% k_{approx} = \frac{1}{2} m \mathbf{v} \cdot \mathbf{v}
%% $$
%% $$
%% \dot{\mathbf{v}} = \begin{pmatrix} kml (c\ddot{\theta} - s\dot{\theta}^2) \\ -kml (s\ddot{\theta} + c\dot{\theta}^2) \end{pmatrix}
%% $$
%% $$
%% k_{approx} = \frac{1}{2} m (kml)^2(\ddot{\theta}^2 + \dot{\theta}^4)
%% $$

%% \section{Linearization}

%% K, I don't actually want to do this because we want to be able to go to any arbitrary position and then squeeze the fingers to hold it there. \\
%% But let's say for shits and giggles we are....\\ \\

%% There are 2 equilbrium points, $\theta=0$ and $\theta=\pi$. We will focus on small variations of $\theta$ about an equilibrium point $\theta_o$, so $\theta = \theta_o+\epsilon$ and $\dot{\theta} = \dot{\epsilon}$ and higher order terms are neglected.

%% \subsection{Pendulum Down}

%% $\cos(\theta) \approx 1$ and $\sin(\theta) \approx \theta$, giving us

%% $$
%% (M+m)\ddot{x} - ml\ddot{\theta} + ml\theta\dot{\theta}^2 = f(t) 
%% $$
%% $$
%% (M+m)\ddot{y} + ml\theta\ddot{\theta} - ml\dot{\theta}^2 - mg = g(t)
%% $$
%% $$
%% (ml^2 + I)\ddot{\theta} + b \dot{\theta} -ml\ddot{x} +ml\ddot{y}\theta + mg\theta= 0
%% $$

%% Now we take the laplace transform, 
%% $$
%% (M+m)s^2X(s) - mls^2\Theta(s) + ml\Theta(s)(s\Theta(s))^2 = F(s) 
%% $$
%% $$
%% (M+m)s^2Y(s) + ml\Theta(s)s^2\Theta(s) - ml(s\Theta(s))^2 - mg = G(s)
%% $$
%% $$
%% (ml^2 + I)s^2\Theta(s) + b s\Theta(s) -mls^2X(s) +mls^2Y(s)\Theta(s) + mg\Theta(s) = 0
%% $$

%% Now we use substitution to eliminate either $X(s)$, $Y(s)$, or $\Theta(s)$ \\

%% Lots o algebra to get transfer functions and plot the pole-zero plots and bode diagrams

%% \section{State Space Representation}

%% For linear state control, we must convert the state equations to state space representation in the form 
%% $$
%% \mathbf{\dot{x}} = \mathbf{Ax} + \mathbf{Bu}
%% $$

%% So now our state is given by,
%% $$
%% \mathbf{x} = \begin{pmatrix}x_1\\x_2\\x_3\\x_4\\x_5\\x_6\end{pmatrix} = \begin{pmatrix}x\\\dot{x}\\y\\\dot{y}\\\theta\\\dot{\theta}\end{pmatrix}
%% $$
%% \noindent
%% Referring to:
%% $$
%% (M+m)\dot{x_2} - ml\dot{x_6} + mlx_5 x_6^2 = f(t) 
%% $$
%% $$
%% (M+m)\dot{x_4} + mlx_5\dot{x_6} - ml x_6^2 - mg = g(t)
%% $$
%% $$
%% (ml^2 + I)\dot{x_6} + b x_6 -ml\dot{x_2} +ml\dot{x_4}x_5 + mgx_5= 0
%% $$

%% Then we get a matrix,
%% $$
%% \frac{d}{dt} \begin{pmatrix}x_1\\x_2\\x_3\\x_4\\x_5\\x_6\end{pmatrix} = \begin{pmatrix}x_1\\x_2\\x_3\\x_4\\x_5\\x_6\end{pmatrix}\begin{pmatrix}x_1\\x_2\\x_3\\x_4\\x_5\\x_6\end{pmatrix}+\begin{pmatrix}x_1\\x_2\\x_3\\x_4\\x_5\\x_6\end{pmatrix}f(t)
%% $$


%% So uh.... do we just plug stuff in trying to get something taht works into our equations....


\end{document}
